\documentclass[twocolumn]{article}			
\usepackage{amsmath}%AMS-LaTeX represent
\usepackage{amsfonts,amssymb}%Fonts
\usepackage[frenchb]{babel}%Francisation
\usepackage{listings}
\usepackage[utf8]{inputenc}
				
%%%%%%%%%%%%%%%%%%%%%%%%%%%%%%%%%%%%%%%%%%%%%%%%%%%%%%%%%%%%%%%%%%%%%%%%%%%%%%%%%%%%%
				% Marges et autres settings pour cheat sheet %
%%%%%%%%%%%%%%%%%%%%%%%%%%%%%%%%%%%%%%%%%%%%%%%%%%%%%%%%%%%%%%%%%%%%%%%%%%%%%%%%%%%%%

\pdfpagewidth 8.5in
\pdfpageheight 11in

\setlength\topmargin{0in}
\setlength\headheight{0in}
\setlength\headsep{0in}
\setlength\textheight{10.5in}
\setlength\textwidth{8in}
\setlength\oddsidemargin{0in}
\setlength\evensidemargin{0in}
\setlength\parindent{0in}
\setlength\parskip{0in}
\renewcommand{\baselinestretch}{1}
\addtolength{\oddsidemargin}{-.8in}
\addtolength{\topmargin}{-.8in}

%%%%%%%%%%%%%%%%%%%%%%%%%%%%%%%%%%%%%%%%%%%%%%%%%%%%%%%%%%%%%%%%%%%%%%%%%%%%%%%%%%%%%
						% Commandes de cheat sheet %
%%%%%%%%%%%%%%%%%%%%%%%%%%%%%%%%%%%%%%%%%%%%%%%%%%%%%%%%%%%%%%%%%%%%%%%%%%%%%%%%%%%%%
\newcommand{\ligne}{ ------------------------------------------------------------------------------------ \newline}
\newcommand{\titre}[1]{\ligne \textbf{#1 }}
\newcommand{\titrej}[1]{\textbf{#1 }}

%%%%%%%%%%%%%%%%%%%%%%%%%%%%%%%%%%%%%%%%%%%%%%%%%%%%%%%%%%%%%%%%%%%%%%%%%%%%%%%%%%%%%
						% Commandes d'informatique %
%%%%%%%%%%%%%%%%%%%%%%%%%%%%%%%%%%%%%%%%%%%%%%%%%%%%%%%%%%%%%%%%%%%%%%%%%%%%%%%%%%%%%

\newcommand{\code}[1]{\texttt{#1}}

\begin{document}

\section{Éclairage}
\begin{description}
   \item[\code{glGenTextures( n, GLuint* texture )}] : Génere $n$ noms de textures.
   \item[\code{glBindTexture( GL\_TEXTURE\_2D, GLuint texture[i] )}] : Définit la texture active.
   \item[\code{glTexImage2D( $\dots$ )}] Charge les données dans l'objet texture sur la carte graphique.
	\item[\code{glEnableClientState( GL\_TEXTURE\_COORD\_ARRAY )}]
	\item[\code{glTexCoordPointer(2, GL\_FLOAT, \textit{stride}, textureCoordDE)}]
	\item[\code{glDisableClientState( GL\_TEXTURE\_COORD\_ARRAY )}]

\end{description}

\subsection{Formules de cone}
\begin{description}
   \item[Direct3D] $\code{effetSpot} = \frac{\cos(\gamma) - \cos(\theta_{\textrm{outer}})}{\cos(\theta_{\textrm{inner}}) - \cos(\theta_{\textrm{outer}})}$ 
   % avec \\
   % cosInner = cosDelta
   % cosOuter = cosInner^(1.01 + c/2)

   \item[OpenGL] $\code{effetSpot} = \cos(\gamma)^{c}$ où \\ $c =$ \code{gl\_LightSource[0].spotExponent}
\end{description}

\subsection{Formules de reflection}
\begin{description}
	\item[\code{composanteAmbiante}] $ = S_{\textrm{ambient}} \cdot M_{\textrm{ambient}} $
	\item[\code{composanteDiffuse}] $ = (\vec{N},\vec{L}) \cdot S_{\textrm{diffuse}} \cdot M_{\textrm{diffuse}} $
	\item[\code{composanteSpeculaire}] $ = (\vec{R},\vec{V}) \cdot S_{\textrm{specular}} \cdot M_{\textrm{specular}} $
\end{description}



\end{document}